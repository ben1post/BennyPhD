
\begin{titlepage}
\begin{center}

% Upper part of the page
\begin{figure}[ht]
\begin{minipage}[b]{0.5\linewidth}
\centering
\includegraphics[width=0.7\textwidth]{./0-titel/jacobs_big.pdf}\\
\end{minipage}
\hspace{0.5cm}
\begin{minipage}[b]{0.5\linewidth}
\centering
\includegraphics[width=0.7\textwidth]{./0-titel/zmt_big.png}\\ 
\end{minipage}
\end{figure}

\vspace{1cm}

\textsc{\Large Jacobs University Bremen}\\[1.5cm]

\textsc{\Large Leibniz Center for Tropical Marine Ecology}\\[1.5cm]

\vspace{2cm}

\textsc{\Large Ph.D. Proposal}\\[0.5cm]

% Title

{ \Large \bfseries Phytoplankton community size-structure in the Atlantic Ocean: 
A trait-based perspective}\\[0.4cm]

\vspace{3cm}
% Author and supervisor
\begin{minipage}{0.4\textwidth}
\begin{flushleft} 
\emph{Author:}\\
Esteban \textsc{Acevedo}
\end{flushleft}
\end{minipage}
\begin{minipage}{0.4\textwidth}
\begin{flushright}
\emph{Supervisors:} \\
Prof. Dr.~Agostino \textsc{Merico}\\
Dr. Gunnar \textsc{Brandt}\\
\emph{Jacobs memeber of the panel:} \\
Prof. Dr.~Matthias \textsc{Ullrich}\\
\emph{External memeber of the panel:} \\
Prof. Dr.~Andreas \textsc{Oschlies}\\
\end{flushright}
\end{minipage}

\vspace{1.5cm}

% Bottom of the page
{\today}


\end{center}

\large 
\textbf{ABSTRACT} \\

\normalsize

In recent years studies on trait-based ecology provided new insights into the mechanisms driving natural species variation. Trait-based ecology links measurable key characteristics that influence the fitness of organisms to ecological or environmental conditions. This new perspective provides a general framework for theoretical studies. I here use this approach to study the dynamics of marine phytoplankton communities. Cell size is the most structuring of all traits that can be used to characterize phytoplankton communities, because it influences many different ecological and physiological processes in these organisms. Trait-based models relying on energy allocation theory and the mechanistic description of trade-offs have been successful in capturing phytoplankton dynamics with a lower degree of complexity than classical ecosystem models based on functional groups. Advances in this emerging field are, however, still required, since most of the research has focused on the bottom-up control of phytoplankton dynamics. Equally important top-down processes, such as the phytoplankton-zooplankton interaction, are less well understood. 
The general aim of this project is to study the processes that drive phytoplankton dynamics in contrasting environmental regions by means of a size-based model. Furthermore, I aim to extend my proposed size-based model to include the description of complex dynamics such as phytoplankton-zooplankton interactions and evolutionary dynamics. My preliminary analysis of cruise data (Atlantic Meridional Transect Programme) confirms the strong relationship between environmental conditions and the phytoplankton community structure on an ocean basin scale. These results are the empirical background to test the size-based model and to study the consequences of environmental and ecological variation for phytoplankton communities.

\end{titlepage}
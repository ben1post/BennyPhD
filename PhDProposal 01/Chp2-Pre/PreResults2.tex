\chapter{Understanding phytoplankton community shifts in the eastern Cariaco basin}

\small {\textbf{}}


\normalsize
\section{Regime Shift in CARIACO data}
General intro sentence" For decades ecologists have been trying to understand how the structure of phytoplankton communities is associated to the environmental conditions, with a particular focus on the causes and consequences of natural variation.

Something about identifying and modeling regime shifts

Talk about the data again

This will be a first! first proper ecological model apart from this Export Flux model only including diatoms \citep{Walsh2002a}


bb


LOOKING AT BIOMASS DYNAMICS, leading over from Intro where i mentioned JP CBN data at the end (ad-lib)
XXX

EXPLAIN THE HYPOTHESES HERE; AND HOW THEY CAN BE TESteD

\section{Methods}


dont really go into depth here, just generally state how things are done, python, odeint, system of ODEs

XXXX

\subsection{Model physics in a tropical coastal setting}
xXXX

Most models built for temperate oceans, since that is where reasearch (and funding) has been most well developed. Fasham NPZD type slabe physics explain.
Why won't this fit well in the Cariaco setting? - mostly due to shallow and comparatively invariable MLD, and nutrient fluxes don't correlate.

HERE I CAN SHOW THE DIFFERENT MODEL RUNS, explain the difference
for this box model needs to get running! This won't be so easy.. so plan ample time my friend!

XXXX

XXXX

XXX


End Methods here

\section{Preliminary Results}


SHOW PROPER RUN, With Biotic components fitting the base run comparatively well, try it!

XXXX (Figure \ref{RegSizeFrac}).



$kkkkkkkkkkkkkkkkkkkkkk$ here the results start, at least the text of it

$^{\circ}$C$^{\circ}$\%$^{\circ}$C$^{\circ}$\%$^{\circ}$C$^{\circ}$\%$^{\circ}$C$^{\circ}$\%


get it, get it


\section{How to complete this project}
XXXXXXXX

essentially just check model physics again, and then
create nice runs, and then go and test the hypotheses, like so and so and so..
XXXXXXX

X

XXX


  % !TEX TS-program = pdflatex
% !TEX encoding = UTF-8 Unicode

% This is a simple template for a LaTeX document using the "article" class.
% See "book", "report", "letter" for other types of document.

\documentclass[twosided,12pt]{report} % use larger type; default would be 10pt

\usepackage[utf8]{inputenc} % set input encoding (not needed with XeLaTeX)
\usepackage{graphicx,natbib,amssymb,amsmath,lineno}
\newcommand{\HRule}{\rule{\linewidth}{0.5mm}}
%%% Examples of Article customizations
% These packages are optional, depending whether you want the features they provide.
% See the LaTeX Companion or other references for full information.


%%% PAGE DIMENSIONS
%\usepackage[a4paper,landscape]{geometry}
\usepackage{geometry} % to change the page dimensions
\geometry{a4paper} % or letterpaper (US) or a5paper or....
% \geometry{margins=2in} % for example, change the margins to 2 inches all round
% \geometry{landscape} % set up the page for landscape
%   read geometry.pdf for detailed page layout information

%\usepackage{graphicx} % support the \includegraphics command and options

% \usepackage[parfill]{parskip} % Activate to begin paragraphs with an empty line rather than an indent

%%% PACKAGES
\usepackage[abs]{overpic}
\usepackage{setspace}
\usepackage{wrapfig}
\usepackage{pifont}
\usepackage{booktabs} % for much better looking tables
\usepackage{array} % for better arrays (eg matrices) in maths
\usepackage{paralist} % very flexible & customizable lists (eg. enumerate/itemize, etc.)
\usepackage{verbatim} % adds environment for commenting out blocks of text & for better verbatim
\usepackage{subfig} % make it possible to include more than one captioned figure/table in a single float
% These packages are all incorporated in the memoir class to one degree or another...
%\usepackage{times}


%%% HEADERS & FOOTERS
\usepackage{fancyhdr} % This should be set AFTER setting up the page geometry
\pagestyle{fancy} % options: empty , plain , fancy
\renewcommand{\headrulewidth}{0pt} % customise the layout...
\lhead{}\chead{}\rhead{}
\lfoot{}\cfoot{\thepage}\rfoot{}

%%% SECTION TITLE APPEARANCE
\usepackage{sectsty}
\allsectionsfont{\sffamily\mdseries\upshape} % (See the fntguide.pdf for font help)
% (This matches ConTeXt defaults)

%%% ToC (table of contents) APPEARANCE
\usepackage[nottoc,notlof,notlot]{tocbibind} % Put the bibliography in the ToC
\usepackage[titles,subfigure]{tocloft} % Alter the style of the Table of Contents
%\renewcommand{\cftsecfont}{\sfamily\mdseries\upshape}
%\renewcommand{\cftsecpagefont}{\sffamily\mdseries\upshape} % No bold!

%%% END Article customizations
\renewcommand{\familydefault}{\sfdefault}
%%% The "real" document content comes below...

%\title{A trait-based modelling approach to phytoplankton community size-structure in the Atlantic Ocean}
%\author{Esteban Acevedo}
%\date{} % Activate to display a given date or no date (if empty),
         % otherwise the current date is printed 

\begin{document}

\pagenumbering{roman}
%
\begin{titlepage}
\begin{center}

% Upper part of the page
\begin{figure}[ht]
\begin{minipage}[b]{0.5\linewidth}
\centering
\includegraphics[width=0.7\textwidth]{./0-titel/jacobs_big.pdf}\\
\end{minipage}
\hspace{0.5cm}
\begin{minipage}[b]{0.5\linewidth}
\centering
\includegraphics[width=0.7\textwidth]{./0-titel/zmt_big.png}\\ 
\end{minipage}
\end{figure}

\vspace{1cm}

\textsc{\Large Jacobs University Bremen}\\[1.5cm]

\textsc{\Large Leibniz Center for Tropical Marine Ecology}\\[1.5cm]

\vspace{2cm}

\textsc{\Large Ph.D. Proposal}\\[0.5cm]

% Title

{ \Large \bfseries Phytoplankton community size-structure in the Atlantic Ocean: 
A trait-based perspective}\\[0.4cm]

\vspace{3cm}
% Author and supervisor
\begin{minipage}{0.4\textwidth}
\begin{flushleft} 
\emph{Author:}\\
Esteban \textsc{Acevedo}
\end{flushleft}
\end{minipage}
\begin{minipage}{0.4\textwidth}
\begin{flushright}
\emph{Supervisors:} \\
Prof. Dr.~Agostino \textsc{Merico}\\
Dr. Gunnar \textsc{Brandt}\\
\emph{Jacobs memeber of the panel:} \\
Prof. Dr.~Matthias \textsc{Ullrich}\\
\emph{External memeber of the panel:} \\
Prof. Dr.~Andreas \textsc{Oschlies}\\
\end{flushright}
\end{minipage}

\vspace{1.5cm}

% Bottom of the page
{\today}


\end{center}

\large 
\textbf{ABSTRACT} \\

\normalsize

In recent years studies on trait-based ecology provided new insights into the mechanisms driving natural species variation. Trait-based ecology links measurable key characteristics that influence the fitness of organisms to ecological or environmental conditions. This new perspective provides a general framework for theoretical studies. I here use this approach to study the dynamics of marine phytoplankton communities. Cell size is the most structuring of all traits that can be used to characterize phytoplankton communities, because it influences many different ecological and physiological processes in these organisms. Trait-based models relying on energy allocation theory and the mechanistic description of trade-offs have been successful in capturing phytoplankton dynamics with a lower degree of complexity than classical ecosystem models based on functional groups. Advances in this emerging field are, however, still required, since most of the research has focused on the bottom-up control of phytoplankton dynamics. Equally important top-down processes, such as the phytoplankton-zooplankton interaction, are less well understood. 
The general aim of this project is to study the processes that drive phytoplankton dynamics in contrasting environmental regions by means of a size-based model. Furthermore, I aim to extend my proposed size-based model to include the description of complex dynamics such as phytoplankton-zooplankton interactions and evolutionary dynamics. My preliminary analysis of cruise data (Atlantic Meridional Transect Programme) confirms the strong relationship between environmental conditions and the phytoplankton community structure on an ocean basin scale. These results are the empirical background to test the size-based model and to study the consequences of environmental and ecological variation for phytoplankton communities.

\end{titlepage}
\begin{titlepage}
\begin{center}

% Upper part of the page
\begin{figure}[ht]
\begin{minipage}[b]{0.5\linewidth}
\centering
\includegraphics[width=0.7\textwidth]{./0-titel/ZMT_Logo_BILDMARKE_rgb_ENG.png}\\ 
\end{minipage}
\hspace{0.5cm}
\begin{minipage}[b]{0.5\linewidth}
\centering
\includegraphics[width=0.7\textwidth]{./0-titel/jacobs_big.pdf}\\
\end{minipage}
\end{figure}

\vspace{1cm}

\text{\Large Leibniz Centre for Tropical Marine Research}\\[1.cm]

\text{\Large Jacobs University Bremen}\\[1.5cm]

\vspace{2cm}

\text{\Huge PhD Proposal}\\[0.5 cm]

% Title

{ \Huge \bfseries Modeling changing phytoplankton communities in the Cariaco Basin, Venezuela}\\[0.4cm]


\vspace{3 cm}
% Author and supervisor
\begin{minipage}{0.4\textwidth}
\begin{flushleft} 
\emph{Author:}\\
Benjamin \MakeUppercase{Post}
\end{flushleft}
\end{minipage}
\begin{minipage}{0.4\textwidth}
\begin{flushright} 
\emph{Dissertation Committee:} \\
Prof. Dr.~Agostino \MakeUppercase{Merico}\\
Prof Dr. Marc-Thorsten \MakeUppercase{Hütt}\\
Dr. Esteban \MakeUppercase{Acevedo-Trejos}\\
Prof Dr.~Andrew D. \MakeUppercase{Barton}\\
\end{flushright}
\end{minipage}

\vspace{2cm}

% Bottom of the page
{August 16, 2019 }

\end{center}

%\newpage

% Abstract Page
\large 
\textbf{Abstract} \\

\normalsize

%OK, still need some work here! 

Driven by anthropogenic influences, the environmental conditions on our planet are changing at an unprecedented rate and so are the communities and organisms inhabiting it. As oceanic primary producers, phytoplankton plays a central role in the global biogeochemical cycles and are a key component of computational models exploring the biogeochemistry of the ocean. The field of phytoplankton modeling has seen great advances in the last 20 years, in particular the “trait-based” approach promises ecologically meaningful descriptions of biodiversity by moving away from treating species explicitly, but instead looking at the way organisms interact with the environment (i.e. their traits). Current modeling approaches are moving towards an explicit treatment of biodiversity to explore the effects on ecosystem function. The CARIACO time-series, located in the Cariaco Basin off the coast of Venezuela, provides long-term observations of the biogeochemistry of a tropical coastal ecosystem, that includes measurements of phytoplankton taxonomic and functional diversity. Over the more than 20 years of data collection, clear trends of warming and a marked shift in the phytoplankton community have been described. As of yet, there have been no comprehensive ecosystem modeling studies performed using this data. The project aims to further our understanding of phytoplankton communities by using computational ecosystem modeling applied to the CARIACO time-series data. In the first study a phytoplankton functional-type (PFT) model will be used to investigate aggregate biomass changes between contrasting environmental regimes. Towards this study I have developed a modeling framework in the programming language Python. The development of this modeling tool as an open-source package will form the basis of a technical publication. This flexible modeling framework will then be used to investigate the relationships between biodiversity and ecosystem function (BDEF) with an explicit treatment of biodiversity and testing the model mechanism against the relationship observed in the CARIACO time-series data. 



  

\end{titlepage}
%\maketitle
\tableofcontents

\onehalfspacing
%\doublespacing
\pagenumbering{arabic}
\chapter{General Introduction}

\section{The ocean, phytoplankton and why it matters}

The complexity of the ocean and its vast ecosystems has fascinated scientists to this day and most likely will continue to do so far into the future. Myriad life forms are embedded in a matrix so far removed from our mostly dry existence on top the earth’s crust. In the ocean, life moves in dilution, and the equivalents of forests and grasslands are hard to spot unless the concentration of tiny phytoplankton is so large, that deep blue turns into a milky green.

The term phytoplankton refers to microscopic marine photosynthetic organisms. These microorganisms form the basis of the oceanic food web and are primary producers of planetary scale, contributing roughly half of the oxygen in our atmosphere through photosynthesis \citep{Field2009}. Phytoplankton consists of mostly single-celled organisms, prokaryotes and eukaryotes from a highly diverse evolutionary background \citep{Falkowski2004a}. This large genetic diversity is accompanied by a remarkable range of survival strategies, biogeochemical roles, shapes and sizes within the polyphyletic phytoplankton (see Figure \ref{FinkelPhySizeRange} for a size comparison). The emergence of such a large range of organisms and the mechanisms sustaining their persistence has been one of the key questions in phytoplankton ecology over the last 50 years \citep{Hutchinson1961}

\begin{figure}
\centering
\includegraphics[trim = 0mm 0mm 0mm 0mm, clip, width=.9\linewidth]{./Chp1-Intro/SIZEphytoComparison_FinkelEtAl2010.png}
\caption[Scheme]{{\small "A comparison of the size range (maximum linear dimension) of phytoplankton
relative to macroscopic objects." from \cite{Finkel2010}}}
\label{FinkelPhySizeRange}
\end{figure}


The distribution of phytoplankton is driven by the complex physical forces that govern ocean currents and the chemistry of the bodies of water the move. The key components are macronutrients (e.g. nitrogen \& phosphorus) and micronutrients (e.g. iron \& cobalt) welling up from the deeper ocean or flushed in from continental sources. Wherever there are sufficient nutrients available within the euphotic zone, the depth where photosynthetically available radiation (PAR) is 1\% of the surface value, planktonic life begins to thrive. Planktonic live is further bound by the mixed layer depth (MLD), which describes the depth of the relatively homogenous surface layer maintained by wind stress and surface heat fluxes in which the largely immotile phytoplankton can grow. The biomass produced is mostly consumed by higher trophic levels and either assimilated or excreted. Another large portion experiences natural mortality and viral lysis. Microbial degradation drives remineralization within the euphotic zone, which fuels regenerated production \citep{Eppley1979}

A small fraction sinks out of the photic layer as fecal or detrital matter to the deeper ocean and an even smaller fraction reaches the sea floor as sediment (roughly 1 \%) and remains there over geological times \citep{Honjo2008}. This process has been termed the biological carbon pump. Carbon sequestered this way is removed from the ocean-atmosphere system for potentially millions of years. Given the projected rise of atmospheric CO$_2$ levels, it is of grave importance to understand how changes in the phytoplankton community at the surface, driven by anthropogenic stressors and climate change, will affect the carbon burial potential of oceanic ecosystems. Ecosystems along continental margins provide a particularly productive habitat, with only 10\% of total ocean surface area covered by continental margins, but 10-15\% of marine primary production and more than 40\% of carbon export to the seabed occurring along coastal lines \citep{Yool2001,Muller-Karger2005}. Phytoplankton growth indirectly feeds a considerable part of earth’s population through fisheries \citep{Stock2017} and even shapes the elemental composition of oceanic water itself \citep{Redfield1958}. Studies have both reported a global declining trend in marine primary production \citep{Boyce2012} and increasing trends in long-term ocean time series \citep{Chavez2011a}. In order to answer questions of how phytoplankton will respond to a changing climate it is necessary to look the diverse phytoplankton community in greater detail. 

\section{Characterizing phytoplankton communities}
From the early days of oceanographic research, scientists have been interested in the microscopic organisms that were floating in samples of sea water. These communities contain many species each and in total there are tens of thousands of species of phytoplankton that inhabit the surface ocean \citep{DeVargas2015}. All phytoplankton species use chlorophyll or bacteriochlorophyll to harvest light as the energy source to fix organic carbon, but there is wide variation in virtually all their other properties \citep{Litchman2008}. In addition to the complex community composition, there are many factors affecting measurements of their bulk properties in the ocean, such as the viral and bacterial community and the influence of diverse grazers, all within the complex three-dimensional physical environment that is the ocean. 

Where earlier phytoplankton ecologists focused on identifying individual species, decoding their phylogeny or growing them in controlled lab cultures, recent research is trying to integrate the insights gained from these approaches and quantify the diversity on higher levels of organization in relation to other properties of the ecosystem. The focus has shifted towards trait diversity both within and across species and within and across phytoplankton groups. In order to scientifically describe this perplexing diversity the concepts of trait-based ecology and functional types have been developed \citep{Tilman2001,McGill2006,Violle2007c}.

\begin{figure}
\centering
\includegraphics[width=0.7\linewidth]{./Chp1-Intro/Fig_litchman2008.png}
\caption[Scheme]{\small{"A typology of phytoplankton functional traits" from \cite{Litchman2008}}}
\label{PhytoTraits}
\end{figure}

{\textbf{Functional types, traits and diversity}}


In the following I will try to clarify the complementary terms of phytoplankton traits, functional traits, and functional types. 

The trait-based approach to phytoplankton ecology has been growing in popularity. Part of the fascination evoked by this term stems from its origin in evolutionarily theory. Over the last three decades, it has been adopted by ecologists trying to understand communities and ecosystems. In this new context, the concept of traits has been stretched far beyond its original meaning, which can lead to some confusion surrounding the scope of trait-based methods \citep{Violle2007c}. In the simplest definition, a trait is a surrogate of organismal performance. In the ecological context this has been expanded to surrogates for the performance of populations, communities and entire ecosystems. This can include ecophysiological traits, life-history traits, demographic traits or response and effect traits of ecosystems (see Figure \ref{PhytoTraits} for a selection of phytoplankton traits). Theoretically, any property of an organism or ecosystem could be defined as a trait, but ideally a trait should be functional. Functional traits are defined by \cite{Violle2007c} as "morpho-, physio- or phenological traits which impact fitness indirectly via their effects on growth, reproduction and survival". An important facet of the trait-based approach is to describe organismal function via trade-offs between traits. For example when competing for multiple nutrients, phytoplankton species are thought to be constrained by trade-offs in their competitive ability for one over another resource \citep{Tilman1990}. 

Phytoplankton are extremely diverse and the trait based approach lends itself to generalizations, as traits and ecological trade-offs can be defined and explored irrespective of species or taxa boundaries \citep{McGill2006}. However, depending on the study and hypotheses to be tested, it can be very helpful to structure the diversity of organisms into distinct groups. Major taxonomic groups of phytoplankton can be classified based on their ecological or biogeochemical roles within the ecosystem \citep{Iglesias-Rodriguez2002,Flynn2015}. The concept of functional groups is not in contrast to a trait-based ecology of phytoplankton, but can be complementary to it. By broadly sampling relevant traits across phytoplankton groups and species, functional types can be defined by functional traits and trade-offs and therefore extend the trait-based approach by another level of organization \citep{Litchman2007d}. An early example is the work of Ramón Margalef. Margalef used observations of important traits, such as sinking rates and nutrient utilization to build the concept called "Margalef's mandala" to organize phytoplankton functional types (PFTs) on a spectrum of nutrient availability versus turbulence \citep{Margalef1978}. 

The terms functional group and functional type are used interchangeably, with functional groups more often referring to the grouping of species and the functional type describing the group as a whole, often as implemented in computational models. In fact, the simplification of the phytoplankton community into functional types has been widely used for the design and interpretation of computational models that try to recreate or make predictions about the biogeochemical cycling, biogeographic distribution, productivity and other ecosystem functions of phytoplankton \citep{Gregg2003,LeQuere2005}. Biogeochemically defined functional types are most often used, as these functional traits can usually be well defined within an ecosystem model. Typical examples of such functional groups are silicifers, which broadly correspond the phylogenetic group of diatoms, and calcifiers, which are usually represented by coccolithophores. Such functional types are always simplifications of the natural pyhtoplankton diversity. Silicoflagellates create silicified skeletons like diatoms, but are often not explicitly included because they rarely dominate modern phytoplankton assemblages. The choice of which functional groups to include in a model can also be driven by biogeography or analytical considerations concerning the measurement instrumentation used for a particular study \citep{IrwinAndrewJ.Finkel2017b}. 

In biodiversity research, the trait-based approach has been readily adopted. It used to be that species diversity (i.e. the number of species) was the most important metric, but now it is functional diversity, which can be described by the variance in the value of a functional trait of the community or ecosystem. It is important to keep in mind that functional types are often composed of many species with a possibly large variance in trait values. Recent research is trying to understand the effects of diversity within functional types and within species \citep{Violle2012,Violle2017a,DesRoches2018}.




\section{Modeling phytoplankton communities}
Given the complexity of the ocean ecosystem, it is necessary to aggregate our knowledge of the many smaller parts into comprehensive ecological models in order to test mechanistic hypotheses and investigate their full-scale implications. Computational models of phytoplankton growth have been developed since the 1940s and have greatly increased in sophistication and complexity since then, co-evolving with the rise in computational resources \citep{Gentleman2002a}. Phytoplankton modelling started with formulations based on the Lotka-Volterra equations of predator-prey dynamics \citep{Fleming1939}. From these relatively simple descriptions models evolved to describe the oceanic physical environment and the ecosystem it contains including multiple trophic levels. Originally developed by John Steele with a model ocean split in two layers, the nutrient-phytoplankton-zooplankton (NPZ) and nutrient-phytoplankton-zooplankton-detritus (NPZD) models succeeded in reproducing the typical annual bloom dynamics observed in the temperate ocean \citep{Steele1958,Evans1988,Fasham1990a}. Further developments have been in more exact physiological descriptions of phytoplankton based in cellular metabolism and energy allocation \citep{Geider1997} and both simple and more complicated ecosystem formulations driven by local and global 3D circulation models \citep{Lacroix2007, Hirata2013}.

However, in their simplified approach, these models unavoidably limit the characterization of a diverse phytoplankton community \citep{Bruggeman2009}. These plankton ecosystem models are typically highly aggregated, such that a single variable determines the response of a diverse assemblage of phytoplankton species \citep{Franks2009}. Implementing a meaningful treatment of biodiversity in ecological models is a key challenge in the field of phytoplankton modeling \citep{Queiros2015}. The most apparent way of implementing this within the framework of established NPZD models is to include multiple equations and state variables for different phytoplankton functional types \citep{LeQuere2005}. For every group that fulfills a distinct ecosystem function, a new set of parameters has to be added, which complicates the model structure and increases computational costs. This somewhat intuitive approach, however, does lead to problems. First and foremost, this is the lack and inherent uncertainty of data from field and culture experiments to constrain functional types. This again leads to the difficulty of validating the model output in light of insufficient information, leading multiple authors to criticize the PFT modeling approach as attempting to "run before we can walk" particularly when used for extrapolating into the future \citep{Anderson2005,Shimoda2016}. 

The current scientific discussion can seem intimidating to an early career scientist, as both the most obvious future directions of ecosystem model design as exemplified by PFT models, as well as very basic assumptions, such as using Monod nutrient uptake kinetics in phytoplankton models, have come under scrutiny \citep{Flynn2010, Smith2014, Hellweger2017a}.
Nevertheless, there are also examples of modeling approaches that show a way forward. To name an alternative to the modeling paradigm discussed so far, there is individual based modeling (IBM). In IBM the phytoplankton are explicitly represented as individual agents, allowing for a diverse and spatially interactive phytoplankton community \citep{Hellweger2009}. However, the computational cost and structural complexity of this approach makes both model construction and interpretation difficult. 

Another approach is to extend traditional NPZD models via moment-based estimation of aggregate properties \citep{Merico2009}. A specific implementation of this is the PhytoSFDM model developed by \citet{Acevedo-Trejos2016}. Instead of modeling multiple size-classes of phytoplankton explicitly, the community is described not only by the biomass, but also by the mean size and size variance. Size is used as a master trait, with size variance being used as a proxy for functional diversity. Trade-offs related to nutrient uptake, grazing and sinking structure the phytoplankton community along the size spectrum as driven by the physical forcing. The model was used to investigate latitudinal diversity gradients in the Atlantic Ocean \citep{Acevedo-Trejos2018}. This approach allows a relatively simple model structure and is computationally efficent, but the PhytoSFDM model structure currently only allows for a single phytoplankton community with a log-normal distribution in cell size. This lends the PhytoSFDM model more to studies of large scale processes and biogeographic patterns rather than to local ecosystem modeling studies, where a more fine-scaled resolution of the phytoplankton community might be required.

A major advancement in the field of phytoplankton modeling was the DARWIN model developed at MIT \citep{Follows2007d}. In the general framework of the DARWIN model, large numbers of phytoplankton types are initialized with equal biomass but with different parameters for the most important traits, namely those related to light harvesting, temperature dependence, and nutrient acquisition. These parameters are chosen stochastically from broad ranges of values, based on laboratory and field data, and constrained by simplified allometric functions describing ecological trade-offs. Different functional types are prescribed in the model via varying nutrient utilization traits (e.g. small phytoplankton that cannot assimilate nitrate as Prochlorococcus analogs). Over multi-annual runs this community self-assembles through ecological competition and physical changes produced by the simulated environment of a global circulation model (GCM). In the random initialization of phytoplankton types, the DARWIN model allows for the emergence and development of diverse phytoplankton communities. This approach of modeling biodiversity has been termed “selection-based” \citep{Follows2011c}. The model framework is continually modified and expanded, for example for exploring the effect of grazing formulations \citep{Prowe2012c}, the biogeography of phytoplankton traits \citep{Barton2013} and the influence of ocean acidification at a global scale \citep{Dutkiewicz2015}. A study of particular interest is the size-structured food-web model component developed by \citet{Ward2012}, because the modeling approach combines the trait-based approach of scaling parameters allometrically along cell size with a PFT modeling approach. Each functional type is assigned a different allometry based on the size ranges and relationships taken from data. Together with the selection-based biodiversity representation that the DARWIN framework provides, this seems to be a promising direction for future models. The model has however only been applied and compared to data at a global scale and the code of this specific implementation is not publicly available. 

The field of phytoplankton modeling provides diverse approaches and no consensus has been reached as to which computational tools and model structures are the most promising. In describing an ecosystem in computational models, a balance needs to be found between complexity and simplicity. The natural ecosystems are invariably very complex, however models that capture too much of the detail are difficult to interpret, and overtly simplistic models risk not capturing the essential dynamics present. Another important aspect of model development, no matter where it lies on the scale of complexity, is model validation using ecological data \citep{RykielJr1996}. In fact model complexity might generally have to be adapted to the scientific question at hand, which often emerges from patterns observed in ecological data. 


\section{The Cariaco Basin and the CARIACO time-series}
\label{CARIACOintro}

%need to explain SST somewhere! im using it later a lot

At the beginning of my PhD I set out to find publicly available ocean time-series data that could be used as the basis for modeling work about phytoplankton diversity. Initially the plan was to choose multiple locations, to compare results from model applications in contrasting environments. Not surprisingly the search did quickly yield results, among the most prominent: the Bermuda Atlantic Time-series Study (BATS) and the Hawaii Ocean Time-Series (HOTS). Quickly the issues of public ocean time-series data became apparent. Links often lead to defunct sites and servers were sporadically maintained. In particular for my application the problem was that the basic physical parameters and bulk properties such as total chlorophyll were readily available, but more specific phytoplankton functional type or taxonomy data was harder to find, if not missing. In particular the type of phytoplankton data that was available differed widely between the stations and would have not allowed for a straightforward comparison. It was only later in my search that I came across the CARIACO time series, an acronym for "CArbon Retention In A Colored Ocean", located in the Cariaco Basin off the coast of Venezuela. The data is available through the University of South Florida (USF) at \href{http://imars.marine.usf.edu/cariaco}{http://imars.marine.usf.edu/cariaco} and includes a wealth of data that was collected since 1995. Most importantly to my purposes, the data included both detailed phytoplankton pigment measurements and taxonomic data of the phytoplankton community at monthly intervals. It was the first ocean time-series with such detailed public phytoplankton data and soon I decided to focus my work on the Cariaco Basin. Previous work specifically using the phytoplankton dataset has to my knowledge been limited to statistical modeling and analysis \citep{Taylor2012,Mutshinda2013a, Mutshinda2013,Pinckney2015,Irwin2015}. Apart from an early study looking at carbon fluxes, where the only resolved functional type in the ecosystem model were diatoms \citep{Walsh2002a}, there has been no study using a mechanistic ecosystem model applied to the CARIACO time-series, allowing for novel applications and insights from my PhD thesis.    

The location of the CARIACO time-series has historically been a site of scientific interest and served as a natural laboratory in biogeochemical research for more than 60 years \citep{Muller-Karger2019a}. It has played important roles in understanding the nutrient stoichiometry of the ocean \citep{Redfield1963}. The CARIACO Ocean Time-Series program was established in 1995 off the coast of Venezuela (10$^\circ$ 30$'$ N, 64$^\circ$ 40$'$ W, see Figure \ref{CARIACO-map}). Located in the south-eastern Caribbean Sea, the Cariaco Basin is a 160 km long and 70 km wide tectonic depression, reaching up to 1400 m in depth. The two deeper parts of the basin are separated by a saddle of 900 m depth, with the time-series mooring located in the eastern part. The entire basin is bound to the west and north by a shallow ridge at 100 m depth, restricting the exchange of deep water with the Caribbean Sea. The restricted circulation and high productivity at the surface resulted in anoxic conditions below 250 m depth within the basin \citep{Richards1956}. The Cariaco Basin is the worlds largest truly marine anoxic basin \citep{Wakeham2012}. The hydrography at the surface is influenced by Guyana and North Equatorial currents that flow into the Caribbean Sea from a south-eastern direction, but this exchange is restricted to the two channels above the 100 m ridge. Observed and modeled horizontal surface water velocities within the basin are relatively weak, indicating a minimal influence of horizontal transport at the mooring site \citep{Alvera-Azcarate2009}. 

\begin{figure}
\centering
\includegraphics[trim = 0mm 0mm 0mm 0mm, clip, width=1.\linewidth]{./Chp1-Intro/CARIACObasinMAP_Bringueetal2018.png}
\caption[Scheme]{\small {"Study area. A. Location of the Cariaco Basin off the Venezuelan coast in the southern Caribbean Sea, with January and June positions of the Intertropical Convergence Zone (ITCZ). B. Location of the CARIACO station in the eastern sub-basin, general bathymetry and local rivers emptying in the basin (bathymetric data from GEBCO\_08 Grid)" from \cite{Bringue2019}}}
\label{CARIACO-map}
\end{figure}



{\textbf{The time-series data}}

The program was established as a joint-project of the Venezuelan Fondo Nacional de Cienca, Tecnolog\'{i}a e Investigac\'{i}on (FONACIT) and the US National Science Foundation (NSF), with the particular interest in creating a time-series of surface ocean biogeochemistry that could be linked to satellite observations and the sedimentation accumulating in the anoxic basin. Since  November 1995 there were 232 core cruises at mostly monthly intervals, in addition to sediment trap and microbial-biogeochemistry process cruises. The full set of measurements and determination methods taken can be found in the manual of methods that was published by the coordinators of the time-series \citep{Astor2013}. Of particular interest to my applications are the detailed nutrient measurements taken using a Niskin bottle sampler and continous flow analysis (see Figure \ref{CARIACOnuts} for contour plots of Nitrate, Phosphate and Silicate), as well as the phytoplankton taxonomy (see Figure \ref{PFTAbundancesTax} for PFT abundances) and high-pressure liquid chromatography (HPLC) measurements (see Figure \ref{TChlAPinckney} for HPLC derived total chlorophyll a ($chl~a$)) also taken at discrete depth intervals over the duration of the time series. 

\begin{figure}
\centering
\includegraphics[trim = 0mm 0mm 0mm 0mm, clip, width=1.\linewidth]{./Chp1-Intro/NUTSatCARIACOAsset811.png}
\caption[Scheme]{\small {Contour plots of Nitrate, Phosphate and Silicate of the CARIACO time series down to 250 m. White line shows depth of MLD. Surface interpolation using R package MBA.surf.}}
\label{CARIACOnuts}
\end{figure}

The general dynamics observed over the course of the time-series are comprehensively discussed in a recent review paper by \citet{Muller-Karger2019a}. In the same publication it was also announced that the time-series was officially ended, due to a lack of funding. However, in the two decades that it existed the time-series data created a detailed description of the biogeochemistry of the Cariaco Basin. Despite the tropical location of the time series, the data reveals a seasonal cycle driven by upwelling along the southern coastline of the Caribbean Sea occuring usually between November and August, as trade winds intensify with the southward migration of the Intertropical Convergence Zone (ICTZ) (see Figure \ref{CARIACO-map}, A). The influx of nutrients provides the basis for high productivity (320 to 628 g C m$^{-2}$ y$^{-1}$) in the surface waters of the basin. This drives vertical export, with 9 to 10 g C m$^{-2}$ y$^{-1}$ reaching the bottom sediments, which amounts to 1-3 \% of primary production (PP) \citep{Muller-Karger2019a}. The waters are inhabited by a diverse community of microorganism, in particular at the oxic-anoxic interface at roughly 250 m depth where novel eukaryotes have been found \citep{Stoeck2003}. 

\begin{figure}
\centering
\includegraphics[trim = 0mm 0mm 0mm 0mm, clip, width=0.7\linewidth]{./Chp1-Intro/IRWIN_F1.large.jpg}
\caption[Scheme]{\small {"Monthly environmental conditions averaged over the upper mixed layer (1, 7, 15, and 25 m depth) from the CARIACO Ocean Time-Series Program: temperature ($^\circ$C), irradiance (mol m$^{–2}$ d$^{–1}$), and nitrate concentration ($\mu$mol L$^{–1}$). The vertical dotted line is drawn at the boundary (January 1, 2004) between the cool and warm periods. The straight lines are linear regressions: temperature = (24.6 $\pm$ 0.3) + (0.09 $\pm$ 0.03)$t$, $R^2$ = 0.05, P \textless 0.005; irradiance = (18.1 $\pm$ 0.9) + (0.05 $\pm$ 0.11)$t$, $R^2$= 0.001, P = 0.65; nitrate = (1.06 $\pm$ 0.14) – (0.045 $\pm$ 0.017)$t$, $R^2$ = 0.04, P = 0.03, where $t$ is time in years since January 1, 1996, errors are one SE, and the shaded region is the 95\% confidence interval on the line. The $R^2$ is very low because of the tremendous interannual variation relative to the trend." from \cite{Irwin2015}}}
\label{CARIACOTaylorTrends}
\end{figure}

Long-term trends show a reduction in upwelling intensity in the time from 2003 to 2013, which has been linked to a weakening trend in the Trade Winds and the north-easterly movement of the Atlantic centroid of the ICTZ \citep{Taylor2012}. This reduction in upwelling is visible in a reduction nitrate concentration and coincides with an increase in temperature across the upper mixed layer (see Figure \ref{CARIACOTaylorTrends}). The change in physical conditions was accompanied by a shift in the biotic community. Phytoplankton bloom intensities were reduced, while phytoplankton diversity and zooplankton densities showed an increasing trend. Interestingly, vertical export remained at a similar level, despite the fact that phytoplankton taxonomy data shows a marked reduction in larger phytoplankton species. \citep{Taylor2012,Pinckney2015}. The increase in zooplankton densities was linked to a collapse in the local sardine fisheries (see Figure \ref{TaylorSHIFTS}). The biogeochemistry of the deeper waters reflects a shift in the biotic community at the surface \citep{Scranton2014}. 

\newpage
\section{Aims of the proposed PhD project}

The marked dynamics captured in the CARIACO time-series provide a wealth of data to explore during my PhD. With the upcoming collaboration with Andrew Barton at the Scripps Institution of Oceanography I will be working under the supervision of a phytoplankton modeling expert who has direct experience of the CARIACO dataset, as he himself worked on a proposal for an ecosystem modeling study in the region that was unfortunately not funded. I am also in contact with James Pinckney and Claudia Benitez-Nelson from the University of South Carolina, who have been heavily involved in the scientific project of CARIACO and have been kind enough to share specific datasets that were not publicly available. The general aim of my PhD Project, as exemplified by the three planned manuscripts, is to build a framework to explore the shifts observed in the phytoplankton community in the Cariaco Basin.

\begin{itemize}
\item The first study will focus on the biomass dynamics observed in the HPLC-derived phytoplankton pigment data. I have built an ecosystem model that resolves multiple nutrients and multiple functional types of both phytoplankton and zooplankton. The goal is to explore the drivers of community composition. Hypothesis of bottom-up and top-down processes (a decrease in upwelling and an increase in zooplankton grazing respectively) driving these changes have been put forward in the literature. The ecosystem model would allow for a mechanistic exploration of the effects of these processes on the phytoplankton community, with a direct comparison to the dynamics observed in the CARIACO data.
\item During the model construction of the first study, I developed a technical concept to implement a flexible PFT model framework in the open-source programing language Python. The code structure allows for any combination and amount of functional types to be added to the model structure, which lends itself very well to the study of biodiversity effects, as well as simply implementing different types of ecosystem models. This structure is already used within the first study, however I aim to fully develop the code as a Python package, to release to the scientific community. Inspired by the PhytoSFDM package developed by \citet{Acevedo-Trejos2016}, the python package will form the basis of a technical manuscript to be submitted to the journal Geoscientific Model Development.
\item As the first study uses a PFT modeling approach, but otherwise does not explore the effects of biodiversity on ecosystem function, the third study will aim to address this issue with the help of the  developed modeling tool. The CARIACO time-series provides a rich dataset to test hypotheses, in particular of the effects of intra- and inter-PFT diversity. The collaboration with Andrew Barton is central to the formulation the specific scientific questions and model implementations. I aim to complete this project in the third year of my PhD. 
\end{itemize}


%\chapter{A trait-based characterization of phytoplankton communities in contrasting environmental regions of the Atlantic Ocean}

\small {\textbf{Manuscript to be submitted to Marine Ecology Progress Series}}

%\subsection*{}
%\textbf{ABSTRACT}: In recent years trait-based ecology studies had been providing new insights on the mechanisms driving natural variation based on measurable key characteristics of organisms. Here we investigate the phytoplankton community size-composition in the Atlantic Ocean using data from the Atlantic Meridional Transect program. We extended the existing knowledge on the distribution of the phytoplankton size composition in the Atlantic, using a larger data set, integrating phytoplankton size composition, nutrients concentrations, and grazers abundance into a trait-based approach. The selected subset was constrain by k-means partitioning and based on the prevailing environmental conditions. Also we studied how the different phytoplankton size fractions respond to different environmental gradients. Our results suggest a linkage of the \textit{in situ} environmental conditions with community size-composition, and regardless of the spatio-temporal conditions. We discuss how the observed patterns of phytoplankton size-fractions in the Atlantic Ocean are coherent with the niche partitioning theory and opposed to the unified neutral theory of biodiversity.

\normalsize
\section{Introduction}
For decades ecologists have been trying to understand how the phytoplankton patterns of community structure are associated to the environmental conditions, with a particular focus on the causes and consequences of natural variations. One of the approaches adopted in this quest is based on observations of key characteristics of organisms, populations or communities. These key characteristics are also called traits \citep{McGill2006, Violle2007}. 

Trait-based ecology aims at developing an understanding and a better predictability of natural communities by linking traits that influence organism performances and fitness with prevailing environmental conditions \citep{McGill2006}. For example, recent investigations suggest that changes in mean trait and trait variance are invariable to different spatial scales, thus stressing the importance of the environmental conditions on trait variation and implying that the within-species and the interspecific variations on natural communities do not add more variation to the trait \citep{Messier2010}. 

Phytoplankton organisms are ideal systems for trait-based approaches. They are relatively simple with well defined ecological niches which are determined by the physical environment, the resource allocation strategies, and the interspecific relationships \citep{Litchman2007}. They have various morphological, physiological, behavioral and life history traits and trade-offs. Among all the known phytoplankton traits, cell-size is probably the best characterizing property of phytoplankton communities because many ecophysiological processes such as nutrient and light utilization and resistance to grazing, are significantly correlated with cell size \citep{Litchman2008,Litchman2010}. \textit{In vivo} and \textit{in situ} observations of a variety of traits are constantly measured due to the global importance of phytoplankton as primary producer, influencing trophic webs and biogeochemical cycles \citep{Falkowski1998}.

Almost every year, since 1995, two scientific cruises have been  crossing the Atlantic ocean from Plymouth (UK) to South America or South Africa, with observations including size fractionated chlorophyll-a, nitrate, phosphate and silicate concentrations, temperature and zooplankton abundances. These cruises are part of the Atlantic Meridional Transect Programme. Given the spatial extent of the transect, which crosses a range of ecosystems from sub-polar to tropical and from euphotic shelf seas and upwelling systems to oligotrophic mid-ocean gyres, the dataset is ideal for studying phytoplankton community structure and the driving processes of size composition at an ocean basin scale. Previous analyses attempted a description of the occurrences of the different size fractions \citep{Maranon2001} without considering a direct influence of the prevailing environmental conditions. A more comprehensive analysis that integrates all the different AMT data with the available phytoplankton community size-fractions is, to our knowledge, still lacking. Therefore, our work intends to uncover the mechanisms shaping the different phytoplankton community structures in regions of contrasting environmental conditions (that is: regions with different nutrient regimes and grazing pressures).

We broaden previous analyses by considering a larger selection of data than any previous study. More specifically, we integrate phytoplankton size-fractions with temperature, various nutrient concentrations, and zooplankton abundances in the attempt of disentangling the relative contribution of bottom-up and top-down processes in shaping phytoplankton size structure.

A first step in our analyses was to pre-structure the selected AMT dataset according to a well established ecological classification of marine data into ocean provinces characterised by different environmental conditions \citep{Longhurst2006}. In the second step, we propose a new classification using the available data of nutrient regimes and grazing, and compare our results with those of \citet{Longhurst2006}. In the last step, we relate the environmental differences and the relative contributions of bottom-up versus top-down processes to the different community size structures in order to highlight the emergent patterns of community compositions and structures at the ocean basin scale.

\section{Methods}

We composed a dataset by selecting a number of observed variables from the Atlantic Meridional Transect (AMT) Program (www.amt-uk.org). The resulting dataset comprises mixed-layer depth values of size fractionated chlorophyll-a, nitrate, phosphate and silicate concentrations, temperature, and zooplankton abundances (as a relative indication of grazing pressure). As mixed layer depth we considered that depth at which a variation of 0.5 $^\circ$C in temperature and of 0.125 in density is observed with respect to surface value (i.e. first value at 5-10 m depth). We obtained a dataset of 410 samples from a total of 9 AMT cruises (from AMT2 to AMT6, AMT10, AMT11, AMT13, AMT14). These cruises took place in April-May or September-October in the years 1996 (AMT2 and AMT3), 1997 (AMT4 and AMT5), 1998 (AMT6), 2000 (AMT10 -AMT11), and 2003 (AMT13 and AMT14). 

The phytoplankton size fractions available were in the range of picoplankton (0.2-2 $\mu$m), nanoplankton (2 -20 $\mu$m), and microplankon ($>$20 $\mu$m). AMT13 and AMT14 measured four size classes (0.2-2, 2-5 5-10, $>$10 $\mu$m). Thus, to be consistent with the three plankton size ranges, we considered the 2-5 and 5-10 $\mu$m size classes as part of the nanoplankton class and the $>$10 $\mu$m class as part of the microplankton class. We checked for the results with and without these data and there was no appreciable difference in the resulting patterns. The three size fractions were normalized, based on the proportion of each size fraction to the total Chl-a concentration.

\begin{wrapfigure}{r}{0.5\textwidth}
\begin{center}
\includegraphics[trim = 30mm 20mm 25mm 20mm, clip, width=1\linewidth]{./Chp2-Pre/amt_mapFINAL2.png}
\end{center}
\caption[Scheme]{\small {The AMT subset of 410 samples used on this study. The dashed lines represent the simplified limits of the Longhurst (2006) ecological provinces.}}
\label{Map}
\end{wrapfigure}

The selected dataset covers temperate, subtropical and tropical regions of the Atlantic ocean (Figure \ref{Map}). According to Longhurst's (\citeyear{Longhurst2006}) classification, ten ecological provinces were sampled by these cruises: four temperate provinces, comprising the North Atlantic Drift (NADR; n=24), the South Subtropical Convergence (SSTC, n=21), the Subantartic Water Ring (SANT, n=14), and the Falkland Island (FKLD; n=52); two subtropical provinces, comprising the North Atlantic Subtropical Gyral (NAST; n=37) and the South Atlantic Subtropical Gyral (SATL, n=129); three tropical provinces, the North Atlantic Tropical Gyral (NATR, n=51), the Eastern Tropical Atlantic (ETRA, n=13) and the Western Tropical Atlantic (WTRA, n=64); and one upwelling region, the Benguela province (BENG, n=5). With this dataset we analyzed the phytoplankton community size composition and implemented a linear mixed effect model (LME) to test if the community size-structure differs among ecological provinces. The ecological provinces were classified by regions, defined as: temperate (NADR, FKLD, SSTC, SANT), tropical (NATR, WTRA, ETRA), subtropical (SATL, NAST) and upwelling (BENG). The model was fitted to the data using the size fractions and the regions as fixed factors, and the ecological provinces as random factors.

We utilized a cluster analysis based on k-means partitioning to alternatively classify the data according to the prevailing environmental conditions. The environmental variables considered were nitrate + nitrite concentration, phosphate concentration, silicate concentration and temperature. We performed a principal component analysis (PCA) of the environmental variables and the normalized size fractions to assess the relative importance of a specific environmental gradient to the phytoplankton size fractions. We quantified the linear relationships between the environmental and the community size-structure. In addition we also quantified the response of the three size fractions to the grazer abundance, but only using data of the cruises AMT3 and AMT5, due to a limited availability of the zooplankton data. These analyses were carried out using R v2.12.2 (The R Foundation for Statistical Computing, 2011).

\section{Results}

\begin{figure}
\centering
\includegraphics[trim = 20mm 30mm 20mm 20mm, clip, width=0.5\linewidth]{./Chp2-Pre/amt_4RegionsTriSizeFrac4.png}
\caption[Scheme]{\small {Phytoplankton community size structure of four ecological provinces in the Atlantic Ocean.The contours correspond to the convex hull of the size-fractions distribution on each province and the symbols denotes the mean values.}}
\label{RegSizeFrac}
\end{figure}

\begin{figure}
\centering
\includegraphics[trim = 0mm 0mm 0mm 0mm, clip, width=0.8\linewidth]{./Chp2-Pre/amt_MeanSDProvinces.png}
\caption[Scheme]{\small {Mean values ($\pm$sd) of three phytoplankton size fractions in ten ecological provinces of the Atlantic Ocean. The symbols indicate the means of the normalized size fractions of: picoplankton (\ding{108}), nanoplankton(\ding{110}) and microplankton (\ding{115}).}}
\label{means}
\end{figure}

Community size-structure is highly variable in different ecological provinces. (Figure \ref{RegSizeFrac}). When the mean size-fraction values of only four selected provinces are compared, an increasing trend towards bigger phytoplankton sizes can be observed from the warmer provinces in the tropics and subtropics to the colder provinces in the Benguela upwelling. In tropical and subtropical waters (such as WTRA and SATL), picoplankton is the most common size class, with only a few occurrences of nano- and microplankton. Temperate provinces such as NADR are characterised by a more heterogeneous distribution of size classes. By contrast, the upwelling province  shows a distribution of size fractions dominated mainly by pico- and microplankton. Provinces located in the temperate regions appear to show a larger cell size variability, as indicated by a broader standard deviation, when compared to provinces located in the tropical regions (Figure \ref{means}). These differences are significant for the picoplankton (LME/anova, df=3, F=6.6315, p=0.0247) and the microplankton size fractions (LME/anova, df=3, F=5.5189,p=0.0368), while they are less significant for the nanoplankton (LME/anova, df=3, F=2.03341, p=0.2108).

\begin{figure}
\includegraphics[trim = 12mm 15mm 10mm 15mm, clip, width=0.5\linewidth]{./Chp2-Pre/amt_clsEnvFINAL4.png}
\put(1,210){\textbf{b)}}
\put(-180,210){\textbf{a)}}
\includegraphics[trim = 20mm 20mm 20mm 20mm, clip, width=0.5\linewidth]{./Chp2-Pre/amt_mapClsEnv3.png}
\caption[Scheme]{\small {Phytoplankton community size structure as function of environmental conditions. a)Distributions of size-classes clustered according to Tropic and Temperate conditions with low, mid and high nutrient concentration, with contours corresponding to the convex hull of the size-fractions distribution on each cluster and symbols denoting the mean values. b)Geographical distribution of the clusters compared with Longhurst classification, with color-coding reflecting the cluster classification.}}
\label{clusters}
\end{figure}

\begin{table}
\centering
\caption[Scheme]{\small {Mean values of environmental data for the different clusters with low mid and high referring to the amount of nutrient concentration.}}
\label{tableclus}
\begin{tabular} {c c c c c}
cluster & $NO_2^-$ + $NO_3^-$ & $PO_4^{3-}$ & $SiO_4^{2-}$ & Temperature \\ \hline
Tropic & 0.150$\pm$0.575 & 0.064$\pm$0.078 & 1.097$\pm$0.575 & 25.299$\pm$2.000 \\
Temperate low & 0.556$\pm$1.102 & 0.112$\pm$0.141 & 0.816$\pm$0.617 & 17.894$\pm$2.191 \\
Temperate mid & 9.027$\pm$3.593 & 0.799$\pm$0.373 & 2.423$\pm$1.375 & 11.925$\pm$2.797 \\
Temperate high & 30.324$\pm$4.549 & 1.336$\pm$0.208 & 4.590$\pm$1.926 & 6.810$\pm$3.435 \\ \hline
\end{tabular}
\end{table}

Our k-means clustering analysis of temperature, nitrate, phosphate, and silicate concentrations on all provinces show that tropical and subtropical regions share the same environmental characteristics (yellow dots in Figure 2.4b and see also Table 2.1), while temperate provinces are different and are categorized into three main regions: temperate low, temperate mid and temperate high (respectively green, blue and red dots in Figure 2.4b, and see also Table 2.1). In summary, we obtained a new classification of the data into four regions, which explains 87.3\,\% of variance. The mean phytoplankton size compositions of the new four clusters show (Figure 2.4a) an increasing trend from the high-temperature, low-nutrient regions (tropic) towards the low-temperature, high-nutrient regions (temperate). A principal component analysis of all data shows highest loadings for temperature (positive loading) and nitrate concentration (negative loading) with respect to the first principal component (Figure \ref{PrinComp}). Furthermore, the picoplankton size fraction positively correlated with temperature while nutrient concentrations are positively correlated with nano- and microplankton size fractions (Figure \ref{PrinComp}).

A regression analysis of all the data shows that phytoplankton size composition varies with the environmental conditions irrespective of temporal changes (see Figure \ref{response1} and Table \ref{stats}). A shift from a picoplankton dominated community towards a nano- and microplankton dominated community occurs at increasing in nutrient concentrations (see Figure \ref{response1}a, \ref{response1}b and \ref{response1}c). By contrast, at increasing temperature the relative proportion of picoplankton increases from about 40\% to about 80\%, while the nano- and microplankton are both reduce by about 50\%.

When restricting our regression analysis to data of the only two cruises (AMT3 and AMT5) that sampled zooplankton,we can observe that an increase in copepods abundance corresponds to a decline in picoplankton from 70$\%$ to less than 40$\%$, to an increase in nanoplankton from 20$\%$ to 60$\%$, and to no appreciable change in microplankton (Figure \ref{response2}).

\begin{figure}
\centering
\includegraphics[trim = 0mm 0mm 0mm 0mm, clip, width=0.9\linewidth]{./Chp2-Pre/amt_PrinComp.png}
\caption[Scheme]{\small {Principal component analysis of environmental conditions and normalized phytoplankton size fractions.}}
\label{PrinComp}
\end{figure}

\begin{figure}
\includegraphics[trim = 0mm 0mm 0mm 15mm, clip, width=0.5\linewidth]{./Chp2-Pre/amt_NO3_bars2.png}
\put(-180,210){\textbf{a)}}
\put(1,210){\textbf{b)}}
\includegraphics[trim = 0mm 0mm 0mm 15mm, clip, width=0.5\linewidth]{./Chp2-Pre/amt_PO4_bars2.png}
\includegraphics[trim = 0mm 0mm 0mm 15mm, clip, width=0.5\linewidth]{./Chp2-Pre/amt_SiO4_bars2.png}
\put(-180,210){\textbf{c)}}
\put(1,210){\textbf{d)}}
\includegraphics[trim = 0mm 0mm 0mm 15mm, clip, width=0.5\linewidth]{./Chp2-Pre/amt_Temp_bars2.png}
\caption[Scheme]{\small {Response of picoplankton, nanoplankton and microplankton size fraction to nitrite+nitrate, phosphate, silicate and temperature. Bars represent mean values and the error bars the standard deviation.}}
\label{response1}
\end{figure}

\begin{figure}
\centering
\includegraphics[trim = 0mm 0mm 0mm 15mm, clip, width=0.6\linewidth]{./Chp2-Pre/amt_zoo_bars2.png}
\caption[Scheme]{\small {Response of picoplankton, nanoplankton and microplankton size fraction to copepod abundance. Bars represent mean values and the error bars the standard deviation.}}
\label{response2}
\end{figure}

\begin{table}
\centering
\caption[Scheme]{\small {Summary statistics of linear fitting for the response of three size fractions to each environmental variable}}
\label{stats}
\begin{tabular} {c c c c c c c c c c }
& \multicolumn{3} {c} {Picoplankton} & \multicolumn{3} {c} {Nanoplankton} & \multicolumn{3} {c} {Microplankton} \\
& slope & p-value & $r^2$ & slope & p-value & $r^2$ & slope & p-value & $r^2$ \\ \hline
$NO_2^-$+$NO_3^-$ &-0.090 &0.002 & 0.908 &0.050 &0.001 & 0.921 &0.040 &0.010 &0.792 \\
$PO_4^{3-}$ &-0.0812 &0.021 &0.711 &0.042 &0.012 &0.777 &0.039 &0.125 &0.354 \\
$SiO_4^{2-}$ &-0.047 &0.085 &0.455 &0.030 &0.044 &0.597 &0.016 &0.247 &0.142 \\ 
Temperature &0.082 &0.001 &0.914 &-0.047 &0.008 &0.812 &-0.035 &0.003 &0.885 \\
Copepods &-0.063 &0.064 &0.520 &0.068 &0.051 &0.567 &-0.004 &0.788 &-0.222\\ \hline
\end{tabular}
\end{table}


\section{Discussion}
We analysed different data from different AMT cruises, covering mainly two seasons (late spring/early summer and autumn) from 1996 to 2003 and from different areas of the Atlantic Ocean. Our results showed patterns of phytoplankton size distributions characterized by the dominance of picoplankton in oligotrophic (SATL) and tropical (e.g. WTRA) waters and by the dominance of larger size classes in nutrient-rich waters (BENG), consistently with \citet{Maranon2000, Maranon2001, Poulton2006, Moreno-Ostos2011, Huete-Ortega2011}. These phytoplankton size compositions resulted to be strongly associated with changes in environmental conditions (Figures 2.6 and 2.7). The consistency of our results with similar previous studies of the Atlantic Ocean but that used either less AMT data than our study \citep{Maranon2000, Maranon2001, Poulton2006} or data from sources different than the AMT cruises \citep{Moreno-Ostos2011, Huete-Ortega2011} suggests that the different phytoplankton size structures observed are robust features of the Atlantic Ocean. 

The differences in size-fractions we found in regions of contrasting environmental characteristics (tropical versus temperate) support the view \citep{Messier2010} that environmental conditions strongly influence the trait distribution (cell size in our case). 

Comparing the results of our classification approach with the one of Longhurst, we note that the classification in ecological regions made by Longhurst make a better contrast of the size-structures in the different ecological provinces. This may be due to the higher temporal and spatial resolution of the environmental data used by Longhurst. However, the increased trend of the mean values is a feature that is well capture by both approaches.

The results we obtained with the clustering technique (Figure \ref{clusters}) suggest that the areas between 30$^\circ$ N and 30$^\circ$ S are characterize by nutrient and temperature regimes favouring communities with higher proportions of picoplankton size classes. The temperate areas laying at the edges of this region are distinguished by colder waters and higher nutrient concentrations and by a wider range of phytoplankton size class. It is well known that stronger seasonal changes in the temperate areas lead to these shifts in the community composition. 

From our regression analyses (Figures \ref{response1} and \ref{response2}) we inferred a strong control of NO$_3^-$+NO$_2^-$ and temperature on all three size fractions. While only pico- and nanoplankton size fractions appear to significantly respond to changes in PO$_4^{3-}$, SiO$_4^{2-}$ and copepod abundance. These strong controls could be explained by the trade-offs between resource acquisition and predation pressure leading to specific size classes dominating the phytoplankton community under given environmental conditions. For example, smaller phytoplankton cell sizes have a competitive advantage over larger phytoplankton under low nutrient, low light and low grazing pressure \citep{Litchman2008, Litchman2010}. There are also a number of important physiological and ecological processes that strongly depend on phytoplankton cell size \citep{Kiorboe1993, Cermeno2008a, Finkel2009a}, including metabolic rates, maximum nutrient uptake rate, nutrient diffusion, light absorption, sinking velocity, trophic interactions and even diversity within taxa, which  is often a log-normal distribution of body size. Our results are therefore consistent with this general "size rules" \citep{Finkel2009a}, although to our knowledge this is the first time that such aspects are observed in data extending across an entire ocean basin and irrespective of the temporal changes.

In summary our size-based analyses substantiates some remarkable feature concerning the variation of a key trait such as cell size at an ocean basin scale, and irrespective of temporal changes. Moreover, these findings are consistent with Baas Becking's tenet "everything is everywhere, but the environment selects" \citep{BaasBecking1934, DeWit2006,O'Malley2007}, over large-scale environmental changes. However, the hypothesis that the prevailing environmental conditions are the major driving forces of the phytoplankton community structure in the Atlantic Ocean should be further investigated using the trait-based modelling approach of  \citet{Bruggeman2007, Merico2009}, a research direction also promoted very recently by \citet{Follows2011}. This modelling approach will further stimulate the development of a theoretical framework for exploring in detail the effects of large-scale environmental differences on planktonic communities.

\section{Acknowledgements}
This study uses data from the Atlantic Meridional Transect Consortium (NER/0\\/5/2001/00680), provided by the British Oceanographic Data Centre and supported by the Natural Environment Research Council. 

\chapter{Understanding phytoplankton community shifts in the eastern Cariaco basin}

\small {\textbf{This is the current state of progress towards the first manuscript}}


\normalsize
\section{Regime shift in the Cariaco Basin}
The CARIACO time-series has been collecting detailed data on the phytoplankton community in the Cariaco Basin from 1995 to 2017 (see Section \ref{CARIACOintro} for a full description). What has been a particular focus of the research based on this data set is the apparent changes in environmental conditions documented in both the physical boundary conditions as well as the biological data. 
As documented by \citet{Taylor2012}

and further investigated by \citet{Pinckney2015}

Interesting thing is that there was this shift in the PhytoplanktonCommunity but apparently no real reduction in Export! (This is in Taylor and Pinckney somehwere)
this would be coherent with Pinckney, no real reduction in biomass, but shifts in the community and towards greater depth, talk about depth of the euphotic zone! also have that data thanks to JP and CBN

\begin{figure}
\centering
\includegraphics[trim = 0mm 0mm 0mm 0mm, clip, width=0.7\linewidth]{./Chp2-Pre/Tayloretal2012_F3.large.jpg}
\caption[Scheme]{\small {"Shifts in phytoplankton community composition and sardine landings from the southeastern Margarita Island fishery. Monthly observations presented as gray symbols. Box and whisker plots depict binned annual variations in diatom (A), dinoflagellate (B), coccolithophorid (C) inventories integrated over the upper 55 m and sardine fishery landings (D) in metric tons. Boxes represent the interquartile range of all observations (25th to 75th percentiles). Internal horizontal lines and whiskers are medians and 10th to 90th percentiles, respectively. Blue and red horizontal lines represent the grand medians of all observations between 1996 and 2004 and between 2005 and 2009, respectively. Data in early and late bins are significantly different in all cases (ANOVA; P < 0.001). [Fishery data are courtesy of L. W. Gonzáles (Universidad de Oriente, Boca de Río, Isla de Margarita, Venezuela); zero values artificially set at 0.5 for plotting purposes.)" from \citet{Taylor2012}}}
\label{PrinComp}
\end{figure}




The term regime shift is actually not very well defined and has been used... \citep{DeYoung2004a}
There are global trends and indications of a regime shift, but methods to identify regime shifts are not well established and have been critically discussed in the literature \citep{Steele2004a, Mantua2004a, Litzow2016a}. To my knowledge no formal exploration of a potential regime shift has been performed with the CARIACO data, therefore the term regime shift is used here to describe the observed changes in the phytoplankton community and physical environment without presupposing a formally defined state shift in the entire ecosystem. 


\begin{figure}
\centering
\includegraphics[trim = 0mm 0mm 0mm 0mm, clip, width=.9\linewidth]{./Chp2-Pre/Pinckneyetal2015_TotChlAcontoursMLD.png}
\caption[Scheme]{\small {Contour plot of HPLC-measured $chl~a$ for the two time periods with full data coverage (January 1996 to October 2000 and July 2006 to December 2010). Light white dots indicate data points. White line shows the depth of the mixed layer. HPLC Data and MLD depth was received from James Pinckney and Claudia Benitez-Nelson.}}
\label{PrinComp}
\end{figure}


Talk about the data again

Also talk about mutshinda et al studies!
Mutshinda study one, bayesian approach: \cite{Mutshinda2013a}

Mutshinda study, same year, environmental factors: \cite{Mutshinda2013}

culminated in the study in PNAS: "Phytoplankton adapt to a changing environment" \cite{Irwin2015}

detailed phytoplankton data has been used in the previous studies for statistical analysis, but suprisingly not yet for ecosystem or trait-based modeling
This will be a first! first proper ecological model apart from this Export Flux model only including diatoms \citep{Walsh2002a}

FUNCTIONAL TYPES STRUCTURE --> explain linkage between Pigment Data that I use and functional diversity measurements (XMoreno et al. 2012X) take this from Pinckney et al. 2015...
Thus, photopigment-based measures offer an efficient way to quantify community or functional diversity (X Moreno et al., 2012 X). (From Pinckney et al 2015) ..> this would also be good to discuss in part 4!

bb


LOOKING AT BIOMASS DYNAMICS, leading over from Intro where i mentioned JP CBN data at the end (ad-lib)
XXX

EXPLAIN THE HYPOTHESES HERE; AND HOW THEY CAN BE TESteD

Main hypothesis, based on top down and bottom up processes:
Top down grazing has great influence on ecosystem, not just from biomass, but also biodiversity aspects \cite{Prowe2012c}.


\section{Methods}

MODEL STRUCTURE: 

- Cyanos as of yet not implemented as nitrogen fixers, given simplicity of model formulation, but actually are present and could be included: \citep{Montes2013}

dont really go into depth here, just generally state how things are done, python, odeint, system of ODEs

COPY METHODS SECTION FROM PhytoSFDM in a way, but with the current model setup
including the equations and allofthat!

SHOW MODEL SCHEMATICS!

XXXX

\small {\textbf{Model physics in a tropical coastal setting}}

xXXX

Most models built for temperate oceans, since that is where reasearch (and funding) has been most well developed. Fasham NPZD type slabe physics explain.
Why won't this fit well in the Cariaco setting? - mostly due to shallow and comparatively invariable MLD, and nutrient fluxes don't correlate.
Problem of nutrient forcing! If MLD driven, nutrients below MLD are highly variable, only below 100m do we get towards a relatively constant N0 and Si0 (can show plots here!)

Moved from slab physics of PhytoSFDM model \citep{Acevedo-Trejos2016} which is based on Fasham \citep{Evans2003,Fasham1990a} to a box model formulation adapted from Tyrrell \citep{Tyrrell1999}
The specific differences are (show equations):

HERE I CAN SHOW THE DIFFERENT MODEL RUNS, explain the difference
for this box model needs to get running! This won't be so easy.. so plan ample time my friend!

XXXX

XXXX

XXX


End Methods here

\section{Preliminary Results}


SHOW PROPER RUN, With Biotic components fitting the base run comparatively well, try it!

XXXX (Figure \ref{RegSizeFrac}).



$kkkkkkkkkkkkkkkkkkkkkk$ here the results start, at least the text of it

$^{\circ}$C$^{\circ}$\%$^{\circ}$C$^{\circ}$\%$^{\circ}$C$^{\circ}$\%$^{\circ}$C$^{\circ}$\%


get it, get it





\chapter {Further work}

\section{Where to go from here}

\subsection{How to complete project 1}

XXXXXXXX

essentially just check model physics again, and then
create nice runs, and then go and test the hypotheses, like so and so and so..
XXXXXXX

X

XXX


\subsection{How to complete project 2}


Just say that this model was the basis for the previous chapter work, and will be for the rest of my PhD, a toolkit for testing ideas with multiple functional types! go towards selection-based models, like DARWIN
and how they allow to change the biodiversity explicitly, to test hypothesis




\section{BDEF - Project No 3}

 HERE I should cite the Tilman and Ptacnik Papers that Esteban recommended, talk about how Biodiversity influences resource use efficiency
 
ALSO BDEF MODEL BY LOREAU: \citep{Loreau1998b}
 
 And then say how the model I am building is actually very well equipped to deal with this kind of 

\begin{figure}[h]
\centering
\includegraphics[trim = 0mm 0mm 0mm 0mm, clip, width=.8\linewidth]{./Chp3-Further/PinckneyDiversityPigmentData.jpg}
\end{figure}
Fig. 5. "Time series contour plot of photopigment diversity index (H′). Data points are indicated by dots on the plot and the white line shows the mixed layer depth." from \citet{Pinckney2015}
%this needs proper formatting still man

\begin{figure}[h]
\centering
\includegraphics[trim = 0mm 0mm 0mm 0mm, clip, width=.8\linewidth]{./Chp3-Further/Abundances.pdf}
\end{figure}
% need to edit in illustrator


XXXXX

\subsection{Method}
XXXX




Again talk shortly about how biodiversity means ecosystem resilience (kinda) and how climate change and anthropogenic stressors will test, if not break the boundaries of the ecosystem resilience. We are still trying to understand the basic connections between the main organisms and functional types in the ocean. Such that we can only guess at what steady state lies behind the boundary, but perhaps we should better never find out.

XXXXX


XXXX

XXXX


\newpage

\section{Time table} 

\begin{figure}[h]
\centering
\includegraphics[trim = 20mm 25mm 20mm 25mm, clip, width=1\linewidth]{./Chp3-Further/Chronogram.pdf}
\end{figure}



\bibliographystyle{meps.bst} 
\begin{footnotesize}
\bibliography{proposal.bib}
\end{footnotesize}


\end{document}

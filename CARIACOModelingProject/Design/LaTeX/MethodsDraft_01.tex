\documentclass[11pt,letterpaper,english]{article}
\usepackage[T1]{fontenc}
\usepackage[latin1]{inputenc}
\setlength\parskip{\medskipamount}
\setlength\parindent{0pt}
\usepackage{amsmath}
%\usepackage{graphicx}
\usepackage{amssymb}
\usepackage{babel}
%use epsfig package for figs
\usepackage{epsfig}
\makeatother
\usepackage{multirow}

%%\usepackage[scaled=0.92]{helvet}
\usepackage{helvet}
\usepackage[sf]{titlesec}
\renewcommand\familydefault{\sfdefault}

%use lgrind to include code listing
%\usepackage{lgrind}

%other formatting stuff
\oddsidemargin 0pt
\flushbottom
\parskip 10pt
\parindent 0pt
\textwidth 465pt
\topmargin 10pt
\textheight 610pt
\renewcommand{\baselinestretch}{1.0}


\begin{document}

% SOME MACROS
\newcommand{\etal}{{\em et al.}}
\newcommand{\ux}{{\underline{x}}}
\newcommand{\tdt}{{t}} 

{\bf {\large A1. CARIACO model equations}} 

The ecosystem model equations are similar to those used in Acevedo-Trejos et al. (2016).
Most significant changes are that multiple phytoplankton and zooplankton functional types have been added, and that the grazing formulation was expanded to include preferential feeding on certain functional types.

Nitrogen $N$ (and Silicate $Si$ for Diatoms) is assimilated by the phytoplankton types $P_i$, which are grazed by several zooplankton types $Z_j$. Mortality of and excretion from plankton, and sloppy feeding by zooplankton contribute to Detritus $D$. The phytoplankton types include Nanoflagellates $P_{n}$, Diatoms $P_{dt}$, Coccolithophorse, $P_{c}$ and Dinoflagellates $P_{dn}$. There are two Zooplankton types split by size class, named Mikrozooplankton $Z_{\mu}$ and Mesozooplankton $Z_{\lambda}$.

ToDo:  
Highlight in the equations how grazing works, selective feeding, explain difference in $R_j$ between zooplankton types

\begin{eqnarray}
\frac{\partial N}{\partial t} & = & 
\kappa \cdot \left(N_{0} - N\right) + 
\delta^{N}_{D} \cdot D -
\sum_{i=1}^{n_P} [\mu_i \cdot U_{i}(N_0,Si_0)\cdot L_i(PAR)\cdot T_i(SST) \cdot P_{i}] 
\nonumber \\
\frac{\partial Si}{\partial t} & = & 
\kappa \cdot \left(Si_{0} - Si\right) 
- \mu_{dt} \cdot U_{dt}(N_0,Si_0) \cdot L_{dt}(PAR)\cdot T_{dt}(SST) \cdot P_{dt}
\nonumber \\
\frac{\partial P_{i}}{\partial t} & = & 
\mu_{i} \cdot U_{i}(N_0,Si_0)\cdot L_{i}(PAR)\cdot T_{i}(SST) \cdot P_{i}
- m_{i} \cdot P_{i}
- \sum_{j=1}^{n_Z} [I^{tot}_j \frac{p^i_{j} \cdot P_{i}} {R_{j}} Z_{j}] -
\frac{v}{M(t)} \cdot P_{i} -
\kappa \cdot P_{i}
\nonumber \\
\frac{\partial Z_{\mu}}{\partial t} & = & 
\delta_Z \cdot I^{tot}_{\mu} \cdot Z_{\mu}-
\mu^{}_{\lambda} \frac{Z_{\mu}}{Z_{\mu}+k_{\lambda}} Z_{\lambda}-
\kappa_{Z} \cdot Z_{\mu} -
m_{\mu} \cdot Z_{\mu} - 
g_{\mu} \cdot Z_{\mu}^{2}
\nonumber \\
\frac{\partial Z_{\lambda}}{\partial t} & = & 
\delta_Z \cdot I^{tot}_{\lambda} \cdot Z_{\lambda}+
\delta_{\lambda} \cdot \mu^{}_{\lambda} \frac{Z_{\mu}}{Z_{\mu}+k_{\lambda}} Z_{\lambda}-
\kappa_{Z} \cdot Z_{\lambda} -
m_{\lambda} \cdot Z_{\lambda} - 
g_{\lambda} \cdot Z_{\lambda}^{2}
\nonumber \\
\frac{\partial D}{\partial t} & = & 
\sum_{j=1}^{n_Z} [(1-\delta_Z) I^{tot}_j \cdot Z_{j}] +
(1-\delta_{\lambda}) \cdot \mu^{}_{\lambda} \frac{Z_{\mu}}{Z_{\mu}+k_{\lambda}} Z_{\lambda}-
\sum_{j=1}^{n_Z} [m_j \cdot Z_{j}] +
\sum_{i=1}^{n_P} [m_i \cdot P_{i}] -
\kappa \cdot D -
\delta^{N}_{D} \cdot D
\nonumber
\end{eqnarray}
 
where:\\
\mbox{} \hspace{.5cm} $N_0=$ Nitrogen concentration right below mixed layer [$\mu M$],\\
\mbox{} \hspace{.5cm} $N=$ Nitrogen concentration above mixed layer [$\mu M$],\\
\mbox{} \hspace{.5cm} $v=$ sinking rate of $P_i$ [$m$ $day^{-1}$],\\
\mbox{} \hspace{.5cm} $M(t)=$ mixed layer depth at time point $t$ [$m$],\\
\mbox{} \hspace{.5cm} $\kappa = \frac{1}{M(t)} \cdot \left(h^{+}(t) + \kappa\right)$ Constant that parameterizes diffusive mixing across the thermocline, \\
\mbox{} \hspace{.5cm} $h^{+}(t) = \max\left(0, \frac{d}{d t} M(t)\right)$ Function that describes entrainment and detrainment of material,\\
\mbox{} \hspace{.5cm} $\delta^N_D=$ Remineralization rate of nitrogen component of detritus $D$ [$\mu M d^{-1}$],\\
\mbox{} \hspace{.5cm} $\mu_i=$Growth rate of phytoplankton type $i$ [$d^{-1}$],\\


\mbox{} \hspace{.5cm} $U_i=\begin{cases}\min\left(\frac{N}{N + U^{N}_i}, \frac{Si}{Si + U^{Si}_i}\right),& \text{if P-type is Diatom}\\\frac{N}{N + U^{N}_i}, & \text{otherwise}\end{cases}$ Nutrient uptake of phytoplankton $i$,\\

\mbox{} \hspace{.5cm} $L_i=\frac{1}{M(t) \cdot k_{w}} \cdot \left(e^{1 - \frac{PAR(t)}{Opt^{I}_i}} + e^{1 - \frac{PAR(t)}{Opt^{I}_i} \cdot e^{- M(t) \cdot k_{w}}}\right)$ Light dependence of  phytoplankton $i$,\\
\mbox{} \hspace{.5cm} $T_i= e^{0.063 \cdot SST}$ Temperature dependence of phytoplankton $i$,\\

\mbox{} \hspace{.5cm} $P_i=$ Biomass of phytoplankton type $i$ [$\mu M N$],\\
\mbox{} \hspace{.5cm} $m_i=$ Mortality/excretion rate for phytoplankton type $i$,\\

\mbox{} \hspace{.5cm} $I^{tot}_j= \mu^{Z}_j⋅\frac{R_{j}}{R_{j} + k^Z_j}$ Total intake of zooplankton type $j$,\\
\mbox{} \hspace{.5cm} $k^Z_j =$ Half saturation constant of zooplankton type $j$,\\
\mbox{} \hspace{.5cm} $R_{j}= \sum_{i} (p_{i j}⋅P_{i})$ Total ressource density of zooplankton type $j$,\\
\mbox{} \hspace{.5cm} $p^i_{j}=$ Feeding preference of zooplankton type $j$ feeding on phytoplankton type $i$,\\
\mbox{} \hspace{.5cm} $R_{\mu}= p^n_{\mu}⋅P_{n} + p^{dn}_{\mu}⋅P_{dn} + p^c_{\mu}⋅P_{c}$ Total ressource density of Mikrozooplankton $Z_{\mu}$,\\
\mbox{} \hspace{.5cm} $R_{\lambda}= p^{dt}_{\lambda}⋅P_{dt} + p^{dn}_{\lambda}⋅P_{dn} + p^c_{\lambda}⋅P_{c}$ Total ressource density of Mesozooplankton $Z_{\lambda}$,\\



\mbox{} \hspace{.5cm} $Z_j=$ Biomass of zooplankton type $j$ [$\mu M N$],\\
\mbox{} \hspace{.5cm} $\delta_{Z}=$ Grazing efficiency of zooplankton on phytoplankton (represents sloppy feeding), \\
\mbox{} \hspace{.5cm} $K_{Z}=\frac{1}{M(t)} \cdot \frac{d}{d t} M(t)$ Mixing term of zooplankton, \\
\mbox{} \hspace{.5cm} $g_{i}=$ Higher order predation on zooplankton (quadratic), \\
\mbox{} \hspace{.5cm} $m_{j}=$ Mortality/excretion rate for zooplankton type $j$,\\




\vspace{.2cm}

{\it {\bf A1.1. Phytoplankton growth:}}\\
\[
\mu_j = \mu_{max_{j}} \gamma_j^T \gamma_j^I \gamma_j^N
\]
where\\
\mbox{} \hspace{.5cm} $\mu_{max_{j}}=$ maximum growth rate of phytoplankton $j$,\\
\mbox{} \hspace{.5cm} $\gamma_j^T=$Modification of growth rate by
temperature for phytoplankton $j$,\\
\mbox{} \hspace{.5cm} $\gamma_j^I=$Modification of growth rate by light for
phytoplankton $j$,\\
\mbox{} \hspace{.5cm} $\gamma_j^N=$Modification of growth rate by nutrients
for phytoplankton $j$.\\

Temperature modification (Fig. \ref{fig-growexp1}a):\\
\[
\gamma_j^T= \frac{1}{\tau_1} (A^T e^{-B(T-T_o)^c} - \tau_2 )
\]
where coefficients $\tau_1$ and $\tau_2$ normalize the maximum
value, and $A,B,T_o$ and $C$ regulate the form of the temperature
modification function. $T$ is the local model ocean temperature.

Light modification (Fig. \ref{fig-growexp1}b):\\
\[
\gamma_j^I= \frac{1}{F_o} (1-e^{k_{par} I} ) e^{-k_{inhib} I}
\]
where $F_{o}$ is a factor controlling the maximum value, $k_{par}$ is the
PAR saturation coefficient and $k_{inhib}$ is the PAR inhibition factor.
$I$ is the local PAR, that has been attenuated through the water column
(including the effects of self-shading).

Nutrient limitation is determined by the most limiting nutrient:
\[
\gamma_j^N = \min(N_i^{lim})
\]
where typically
$N_i^{lim}=\frac{N_i}{N_i+\kappa_{N_{ij}}}$
(Fig. \ref{fig-growexp1}c) and $\kappa_{N_{ij}}$ is the half saturation constant of nutrient $i$ for phytoplankton $j$.

When we include the nitrogen as a potential limiting nutrient (EXP2) we 
modify $N_i^{lim}$ to take into account the uptake inhibition caused by ammonium:
\begin{align*}
N_N^{lim} &= \frac{NO_2}{NO_2+\kappa_{IN}} e^{-\psi NH_4}
+\frac{NH_4}{NH_4 + \kappa_{NH4}}  && \text{(nsource=1)} \\
N_N^{lim} &= \frac{NH_4}{NH_4 + \kappa_{NH4}}  && \text{(nsource=2)} \\
N_N^{lim} &= \frac{NO_3 + NO_2}{NO_3+NO_2+\kappa_{IN}} e^{-\psi NH_4}
+\frac{NH_4}{NH_4 + \kappa_{NH4}}  && \text{(nsource=3)}
\end{align*}
where $\psi$ reflects the inhibition and $\kappa_{IN}$ and $ \kappa_{NH4}$
are the half saturation constant of $IN=NO_3+NO_2$ and $NH_4$ respectively.

\vspace{.2cm}

{\it {\bf A1.2. Zooplankton grazing:}}\\
\[
 g_{jk} =g_{max_{jk}} \frac{\eta_{jk} P_j}{A_k} \frac{A_k}{A_k+\kappa^P_k}
\]
where\\
\mbox{} \hspace{.5cm} $g_{max_{jk}}=$ Maximum grazing rate of zooplankton $k$ on
phytoplankton $j$,\\
\mbox{} \hspace{.5cm} $\eta_{jk}=$ Palatibility of plankton $j$ to zooplankton $k$,\\
\mbox{} \hspace{.5cm} $A_k=$ Palatibility (for zooplankton $k$) weighted total phytoplankton concentration,\\
\mbox{} \hspace{1.1cm} $=\sum_j [\eta_{jk} P_j$] \\
\mbox{} \hspace{.5cm} $\kappa^P_k=$Half-saturation constant for grazing of zooplankton $k$,\\


\vspace{.2cm}

{\it {\bf A1.3. Inorganic nutrient Source/Sink terms:}}\\
$S_{N_i}$ depends on the specific nutrient, and includes the remineralization
of organic matter, external sources and other non-biological transformations:
\begin{eqnarray}
S_{PO4} & = & r_{DOP} DOP + r_{POP} POP \nonumber \\
S_{Si}  & = & r_{POSi} POSi \nonumber \\
S_{FeT} & = &  r_{DOFe} DOFe + r_{POFe} POFe -c_{scav} Fe' + \alpha F_{atmos} \nonumber \\
S_{NO3} & = &  \zeta_{NO3} NO_2 \nonumber \\
S_{NO2} & = &  \zeta_{NO2} NH4 - \zeta_{NO3} NO_2 \nonumber \\
S_{NH4} & = &  r_{DON} DON + r_{PON} PON - \zeta_{NO2} NH_4 \nonumber
\end{eqnarray}

where:\\
\mbox{} \hspace{.5cm} $r_{DOM_i}=$Remineralization rate of DOM for element
$i$, here P, Fe, N,\\
\mbox{} \hspace{.5cm} $r_{POM_i}=$Remineralization rate of POM for element
$i$, here P, Si, Fe, N,\\
\mbox{} \hspace{.5cm} $c_{scav}=$scavenging rate for free iron,\\
\mbox{} \hspace{.5cm} $Fe'=$free iron, modelled as in Parekh et al (2004), \\
\mbox{} \hspace{.5cm} $alpha=$solubility of iron dust in ocean water, \\
\mbox{} \hspace{.5cm} $F_{atmos}=$atmospheric deposition of iron dust on surface of model ocean,\\
\mbox{} \hspace{.5cm} $\zeta_{NO3}=\zeta_{NO3}^0(1-I/I_0)_+=$oxidation rate of NO$_2$ to NO$_3$,\\
\mbox{} \hspace{.5cm} $\zeta_{NO2}=\zeta_{NO2}^0(1-I/I_0)_+=$oxidation rate of NH$_4$ to NO$_2$ (is photoinhibited),\\
\mbox{} \hspace{.5cm} $I_0=$critical light level below which oxidation occurs,\\

The remineralization timescale $r_{DOi}$ and $r_{POi}$ parameterizes the break
down of organic matter to an inorganic form through the microbial loop.


{\it {\bf A1.3.1 Fe chemistry:}}\\
\begin{eqnarray}
Fe' & = & FeT - FeL \nonumber \\
FeL & = & L_{tot} -
  \frac{ L_{stab} (L_{tot} - FeT) - 1
        +\sqrt{(1 - L_{stab} (L_{tot} - FeT))^2 + 4 L_{stab} L_{tot}}}
     {2 L_{stab}} \nonumber
\end{eqnarray}
($Fe'$ may be constrained to be less than $Fe'_{max}$ while preserving $FeT$).

 
{\it {\bf A1.4 DOM and POM Source terms:}}\\
$S_{DOM_i}$ and $S_{POM_i}$ are the sources of dissolved and particulate
organic detritus arising from mortality, excretion and sloppy feeding of the
plantkon. We simply define that a fixed fraction $\lambda_m$ of the the
mortality/excretion term and the non-consumed grazed phytoplankton
($\lambda_g$) go into the dissolved pool and the remainder into the particulate
pool. 
\begin{eqnarray}
S_{DOM_i} & = & \sum_{j} [\lambda_{mp_{ij}} m^p_j P_j M_{ij}] 
             + \sum_{k} [\lambda_{mz_{ik}} m^z_k Z_{ik}]
             + \sum_{k} \sum_{j} [\lambda_{g_{ijk}} (1-\zeta_{jk})
                                        g_{ij} M_{ij} Z_k ]
\nonumber \\
S_{POM_i} & = & \sum_{j} [(1-\lambda_{m_{ij}}) m^p_j P_j M_{ij}]
             + \sum_{k} [(1-\lambda_{mz_{ik}}) m^z_k Z_{ik}]
             + \sum_{k} \sum_{j} [(1-\lambda_{g_{ijk}}) (1-\zeta_{jk})  
                                       g_{ij} M_{ij} Z_k ]
\nonumber
\end{eqnarray}


\newcommand{\pcm}[1]{P^C_{m#1}}
\newcommand{\pcmax}[1]{P^C_{\textrm{MAX}#1}}
\newcommand{\pcarbon}{P^C}
\newcommand{\chltoc}{\theta}
\newcommand{\chltocmax}{\theta^{\textrm{max}}}
\newcommand{\chltocmin}{\theta^{\textrm{min}}}
\newcommand{\alphachl}{\alpha^{\textrm{Chl}}}
\newcommand{\mQyield}{\mathit{mQ}^{\textrm{yield}}}
\newcommand{\RPC}{R^{PC}}
\newcommand{\phychl}{\mathit{Chl}}
\newcommand{\aphychlave}{A^{\mathrm{phy}}_{\mathrm{Chl,ave}}}

{\it {\bf A1.4 Geider light limitation model:}}\\
The phytoplankton growth rate is given by the carbon-specific photosynthesis rate
(rate of carbon synthesized per carbon present),
\[
  \mu_j = \pcarbon_j
\]
The carbon-specific photosynthesis rate
\[
  \pcarbon_j = \pcm{,j} \begin{cases}
     1 - e^{-\alphachl_j I \chltoc_j/\pcm{,j}} & \text{if }I>0.1 \\
     0                                         & \text{otherwise}
   \end{cases}
\]
depends on the carbon-specific, light-saturated photosynthesis rate
\[
  \pcm{,j}=\pcmax{j} \gamma^N_j \gamma^T_j
\]
and the Chl $a$ to carbon ratio
\[
  \chltoc_j = \left[ \frac{\chltocmax_j}
                   {1 + \chltocmax_j \alphachl_j I / (2 \pcm{,j})}
		   \right]^{\chltocmax_j}_{\chltocmin_j}
\]

The chlorophyll concentration is
\[
  \phychl_j=P_j \RPC_j \chltoc_j
\]

The light limitation factor can be diagnosed
\[
  \gamma^I_j=\pcarbon_j/\pcm{,j}
\]

\[
  \alphachl_j = \mQyield_j \aphychlave
\]

Parameters:\\
\begin{tabular}{@{\qquad}r@{}l}
 $\pcmax{j}    ={}$& Maximum C-spec.\ photosynthesis rate at reference temperature of phytoplankton $j$\\
 $\chltocmax_j ={}$& Maximum Chl a to C ratio if phytoplankton $j$\\
 $\RPC_j       ={}$& Carbon to phosphorus (!) ratio of phytoplankton $j$\\
 $\alphachl_j  ={}$& Chl a-specific initial slope of the photosynthesis-light curve\\
 $\mQyield_j   ={}$& slope of the photosynthesis-light curve per absorption\\
 $\aphychlave  ={}$& absorption ($m^{-1}$) per mg Chl a
\end{tabular}



\newcommand{\Ptot}{P_{\mathrm{tot}}}

{\it {\bf A2 Diagnostics:}}\\
Total phytoplankton biomass:
\[
  \Ptot = \sum_j P_j
\]

\begin{tabular}{llll}
 name & definition && units \\
\hline
 \texttt{PhyTot  } & $\Ptot$                                                        && $\mu\mathrm{M\,P}$ \\
 \texttt{PhyGrp1 } & Total biomass of small phytoplankton with $\texttt{nsrc}=1$    && $\mu\mathrm{M\,P}$ \\
 \texttt{PhyGrp2 } & Total biomass of small phytoplankton with $\texttt{nsrc}=2$    && $\mu\mathrm{M\,P}$ \\
 \texttt{PhyGrp3 } & Total biomass of small phytoplankton with $\texttt{nsrc}=3$    && $\mu\mathrm{M\,P}$ \\
 \texttt{PhyGrp4 } & Total biomass of large non-diatoms                             && $\mu\mathrm{M\,P}$ \\
 \texttt{PhyGrp5 } & Total biomass of diatoms                                       && $\mu\mathrm{M\,P}$ \\
 \texttt{PP      } & Primary production                                             && $\mu\mathrm{M\,P}\, \mathrm{s}^{-1}$ \\
 \texttt{Nfix    } & Nitrogen fixation                                              && $\mu\mathrm{M\,N}\, \mathrm{s}^{-1}$ \\
 \texttt{PAR     } & Photosynthetically active radiation                            && $\mu\mathrm{Ein}\, \mathrm{m}^{-2}\,\mathrm{s}^{-1}$ \\
 \texttt{Rstar01 } & $R^*_{\mathrm{PO4}}$ of Phytoplankton species \#1, \dots       && $\mu\mathrm{M\,P}$ \\
 \texttt{Diver1  } & Number of species with $P_j > 10^{-8}\,\mu\mathrm{M\,P}$       & where $\Ptot>10^{-12}$ \\
 \texttt{Diver2  } & Number of species with $P_j > 0.1\%\,  \Ptot$                  & where $\Ptot>10^{-12}$ \\
 \texttt{Diver3  } & Number of species that constitute 99.9\% of $\Ptot$            & where $\Ptot>10^{-12}$ \\
 \texttt{Diver4  } & Number of species with $P_j > 10^{-5} \cdot \max\limits_j P_j$ & where $\Ptot>10^{-12}$ \\
\end{tabular}





\thispagestyle{empty}

\section*{PhytoMFTM model parameters (preliminary)}

\noindent
\begin{tabular}{llllllllll}

  symbol               & variable   & description & units   & value      & source\\
  \hline
  \multicolumn{3}{l}{Physical parameters:}\\
  \hline
  $\kappa$ & kappa & diffusive mixing constant & [$m$ $day^{-1}$] & 0.1/0.01 & [Fasham, 1990/1993]\\
  $\delta^{N}_{D}$ & deltaD\_N  & remineralization rate & [$day^{-1}$] & 0.05 & [Fasham, 1990]\\
  $k_w$ & kw & light attenuation coefficient & [$m^{-1}$] & 0.2 & [Edwards \& Brindley 1996]\\                           
  \multicolumn{3}{l}{affecting phytoplankton:} \\
  $v$ & v &  phytoplankton sinking constant & $[m day^{-1}]$ & 0.04 & [Edwards \& Brindley 1996]\\                      
  $I_{opt}$ & OptI & optimum irradiance & [$E$ $m^{-2}$ $day^{-1}$] & 30 & [Acevedo-Trejos, 2015]\\
  \\
  \hline
  \multicolumn{3}{l}{Phytoplankton parameters:}\\
  \hline
  $mo_P$ & moP &  mortality/excretion constant & [$day^{-1}$] & 0.09 & [Fasham, 1990]\\
  \\
  \multicolumn{3}{l}{functional type specific:} \\
  $P_{dt}$ & pt1  & Diatoms\\
  \hline
  $\Delta^{dt}_{Si}$ & pt1\_ratioSi & nitrogen to silicate ratio & [$\mu M Si$ $\mu M N^{-1}$] & 1.12 & [Brzezinski, 1985]  \\
  $K^{dt}_{Si}$ & pt1\_K\_Si & half-saturation constant of Si uptake & [$\mu M Si$] & 2 & [Kristiansen et al. 2000]\\
  $U^{dt}_N$ & pt1\_U\_N & half-saturation constant of N uptake & [$\mu M N$] &  0.446 & [Litchman et al. 2007]\\
  $\mu^{dt}_P$ & pt1\_muP & growth rate & [$day^{-1}$] & 1.5 & [Litchman et al. 2007]\\
  \\
  $P_{c}$ & pt2 & Coccolithophores\\
  \hline
  $U^{c}_N$ & pt2\_U\_N & half-saturation constant of N uptake & [$\mu M N$]  &  0.265 & [Litchman et al. 2007]\\
  $\mu^{c}_P$ & pt2\_muP  & growth rate & [$day^{-1}$]& 1.1 & [Litchman et al. 2007] \\
  \\
  $P_{dn}$ & pt3 & Dinoflagellates\\
  \hline
  $U^{c}_N$ & pt3\_U\_N & half-saturation constant of N uptake & [$\mu M N$]  &  0.009 & [Litchman et al. 2007]\\
  $\mu^{c}_P$ & pt3\_muP  & growth rate & [$day^{-1}$] & 0.6 & [Litchman et al. 2007]\\
  \\
  $P_{n}$ & pt4 & Nanoflagellates\\
  \hline
  $U^{n}_N$ & pt4\_U\_N & half-saturation constant of N uptake & [$\mu M N$]  &  0.045 & [Litchman et al. 2007]\\
  $\mu^{n}_P$ & pt4\_muP  & growth rate & [$day^{-1}$] & 1.7 & [Litchman et al. 2007]\\

  \\
  \hline
  \multicolumn{3}{l}{Zooplankton parameters:} \\
  \hline
  $mo_Z$ & moZ &  mortality/excretion constant & [$day^{-1}$] & 0.0125 & [Prowe et al. 2012]\\
  $\delta_Z$ & deltaZ & assimilation coefficient of grazing on $P_i$ & [-] & 0.75 & [Fasham, 1990]\\
  $\delta_{\lambda}$ & deltaLambda & assimilation coefficient of $Z_{\lambda}$ grazing on $Z_{\mu}$ & [-] & 0.75 & [Fasham, 1990]\\ 
  $\mu_{\lambda}$ & muIntGraze & maximum rate of $Z_{\lambda}$ grazing on $Z_{\mu}$ & [$day^{-1}$] & 0.05 & [?]\\
  $k_{\lambda}$ & kIntGraze & half-saturation constant of $Z_{\lambda}$ grazing on $Z_{\mu}$ & [$\mu M N$] & 0.5 & [?]\\ 
  \\
  $Z_{\mu}$ & zt1 & Mikrozooplankton\\
  \hline
  $\mu^{\mu}_Z$ & zt1\_muZ & maximum rate of grazing on $P_i$ & [$day^{-1}$] & 0.1 & [Prowe et al. 2012]\\ 
  $k^{\mu}_P$ & zt1\_Kp  & half-saturation constant of grazing on $P_i$ & [$\mu M N$] & 0.5 & [Prowe et al. 2012]\\
  $g_{\mu}$ & zt1\_pred & higher order predation on $Z_{\mu}$ & [$day^{-1}$] & 0.01 & [?]\\  
  \\
  $Z_{\lambda}$ & zt2 & Mesozooplankton\\
  \hline
  $\mu^{\lambda}_Z$ & zt2\_muZ & maximum rate of grazing on $P_i$ & [$day^{-1}$] & 0.1 & [Prowe et al. 2012]\\ 
  $k^{\lambda}_P$ & zt2\_Kp  & half-saturation constant of grazing on $P_i$ & [$\mu M N$] & 0.5 & [Prowe et al. 2012]\\
  $g_{\lambda}$ & zt2\_pred & higher order predation on $Z_{\lambda}$ & [$day^{-1}$] & 0.01 & [?]\\
  \\
  
  \hline
  \hline
\end{tabular}
\\\\\\\\
Feeding preferences: \\
\noindent
\begin{tabular}{l|l|l|l|l}
 
  & $P_{dt}$ & $P_{c}$ & $P_{dn}$ & $P_{n}$\\
\hline
$Z_{\mu}$ & 0 & 1 & 1 & 1\\
\hline
$Z_{\lambda}$ & 1 & 1 & 1 & 0\\
\hline
\end{tabular}
\\
where number is $p^i_j$ denoting feeding preference of $Z_j$ grazing on $P_i$


\end{document} 
